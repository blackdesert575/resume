%------------------------
% Resume in Latex
% Author : Harshibar
% Based off of: https://github.com/jakeryang/resume
% License : MIT
%------------------------

\documentclass[letterpaper,11pt]{article}


\usepackage{latexsym}
\usepackage[empty]{fullpage}
\usepackage{titlesec}
\usepackage{marvosym}
\usepackage[usenames,dvipsnames]{color}
\usepackage{verbatim}
\usepackage{enumitem}
\usepackage[hidelinks]{hyperref}
\usepackage{fancyhdr}
\usepackage[english]{babel}
\usepackage{tabularx}
% only for pdflatex
% \input{glyphtounicode}

% chinese fonts
\usepackage[UTF8,scheme=plain,fontset=none]{ctex}
\setCJKmainfont{AR PL UMing TW}
\setCJKsansfont{AR PL UMing TW}
\setCJKmonofont{AR PL UMing TW}


% fontawesome
\usepackage{fontawesome5}

% fixed width
\usepackage[scale=0.90,lf]{FiraMono}

% light-grey
\definecolor{light-grey}{gray}{0.83}
\definecolor{dark-grey}{gray}{0.3}
\definecolor{text-grey}{gray}{.08}

\DeclareRobustCommand{\ebseries}{\fontseries{eb}\selectfont}
\DeclareTextFontCommand{\texteb}{\ebseries}

% custom underilne
\usepackage{contour}
\usepackage[normalem]{ulem}
\renewcommand{\ULdepth}{1.8pt}
\contourlength{0.8pt}
\newcommand{\myuline}[1]{%
  \uline{\phantom{#1}}%
  \llap{\contour{white}{#1}}%
}


% custom font: helvetica-style
\usepackage{tgheros}
\renewcommand*\familydefault{\sfdefault} 
%% Only if the base font of the document is to be sans serif
\usepackage[T1]{fontenc}


\pagestyle{fancy}
\fancyhf{} % clear all header and footer fields
\fancyfoot{}
\renewcommand{\headrulewidth}{0pt}
\renewcommand{\footrulewidth}{0pt}

% Adjust margins
\addtolength{\oddsidemargin}{-0.5in}
\addtolength{\evensidemargin}{0in}
\addtolength{\textwidth}{1in}
\addtolength{\topmargin}{-.5in}
\addtolength{\textheight}{1.0in}

\urlstyle{same}

\raggedbottom
\raggedright
\setlength{\tabcolsep}{0in}

% Sections formatting - serif
% \titleformat{\section}{
%   \vspace{2pt} \scshape \raggedright\large % header section
% }{}{0em}{}[\color{black} \titlerule \vspace{-5pt}]

% TODO EBSERIES
% sans serif sections
\titleformat {\section}{
    \bfseries \vspace{2pt} \raggedright \large % header section
}{}{0em}{}[\color{light-grey} {\titlerule[2pt]} \vspace{-4pt}]

% only for pdflatex
% Ensure that generate pdf is machine readable/ATS parsable
% \pdfgentounicode=1

%-------------------------
% Custom commands
\newcommand{\resumeItem}[1]{
  \item\small{
    {#1 \vspace{-1pt}}
  }
}

\newcommand{\resumeSubheading}[4]{
  \vspace{-1pt}\item
    \begin{tabular*}{\textwidth}[t]{l@{\extracolsep{\fill}}r}
      \textbf{#1} & {\color{dark-grey}\small #2}\vspace{1pt}\\ % top row of resume entry
      \textit{#3} & {\color{dark-grey} \small #4}\\ % second row of resume entry
    \end{tabular*}\vspace{-4pt}
}

\newcommand{\resumeSubSubheading}[2]{
    \item
    \begin{tabular*}{\textwidth}{l@{\extracolsep{\fill}}r}
      \textit{\small#1} & \textit{\small #2} \\
    \end{tabular*}\vspace{-7pt}
}

% \newcommand{\resumeProjectHeading}[2]{
%     \item
%     \begin{tabular*}{\textwidth}{l@{\extracolsep{\fill}}r}
%       #1 & {\color{dark-grey}} \\
%     \end{tabular*}\vspace{-4pt}
% }
\newcommand{\resumeProjectHeading}[2]{
    \item
    \begin{tabular*}{\textwidth}{l@{\extracolsep{\fill}}r}
      #1 & {\color{dark-grey} #2} \\  % Include #2 for the date range
    \end{tabular*}\vspace{-4pt}
}


% \newcommand{\resumeSubItem}[1]{\resumeItem{#1}\vspace{-4pt}}
% \newcommand{\resumeSubItem}[1]{\hspace{2em} \resumeItem{#1}\vspace{-4pt}}
% \newcommand{\resumeSubItem}[1]{
%   \item[] \hspace{2em} \small{#1 \vspace{-1pt}}
% }
\newcommand{\resumeSubItem}[1]{
  \begin{itemize}
    \item \small{#1 \vspace{-1pt}}
  \end{itemize}
}


\renewcommand\labelitemii{$\vcenter{\hbox{\tiny$\bullet$}}$}

% CHANGED default leftmargin  0.15 in
\newcommand{\resumeSubHeadingListStart}{\begin{itemize}[leftmargin=0in, label={}]}
\newcommand{\resumeSubHeadingListEnd}{\end{itemize}}
\newcommand{\resumeItemListStart}{\begin{itemize}}
\newcommand{\resumeItemListEnd}{\end{itemize}\vspace{0pt}}

\color{text-grey}

%-------------------------------------------
%%%%%%  RESUME STARTS HERE  %%%%%%%%%%%%%%%%%%%%%%%%%%%%

\setlength{\footskip}{5pt}
\begin{document}

%----------HEADING----------
\begin{center}
    \textbf{\Huge Joe (永鴻) 黃} \\ \vspace{5pt}    
    \hspace{1pt} \faEnvelope \hspace{2pt} \texttt{\href{mailto:yunghunghuang984@gmail.com}{yunghunghuang984@gmail.com}} \hspace{1pt} $|$
    \hspace{1pt} \faGithub \hspace{2pt} \texttt{\href{https://github.com/blackdesert575}{blackdesert575}} \hspace{1pt} $|$
    %\hspace{1pt} \faYoutube \hspace{2pt} \texttt{harshibar} \hspace{1pt} $|$
    \hspace{1pt} \faMapMarker* \hspace{2pt}\texttt{Taiwan (R.O.C) Taipei}
    \\ \vspace{-3pt}
\end{center}

%-----------EXPERIENCE-----------
\section{工作經驗}
  \resumeSubHeadingListStart

\resumeSubheading
  {System / DevOps Engineer}{Jun. 2025 -- Present}
  {雷技資訊科技有限公司}{}
  \resumeItemListStart
    \resumeItem{在\textbf{Linux}終端機環境中工作。}
    \resumeItem{使用Pakcer 構築與維護以\textbf{Rocky Linux}作為基底的EC2 機器映像檔(AMI)}
    \resumeItem{主要在台灣工程團隊內,維運\textbf{AWS 與地端(on-premise)基礎設施},並實際導入與使用 以 Git 為核心的 \textbf{Infrastructure as Code 工作流程(Terraform / Terragrunt / Atlantis)}}
    \resumeItem{協助審查與驗證既有\textbf{技術文件與基礎設施流程},並辨識跨區域之間的流程與標準不一致問題。}
    \resumeItem{維護並實作 透過 Ansible 進行的系統初始化自動化(包含 Linux 設定、Java 執行環境、Nginx、RabbitMQ、NFS 部署)}
    \resumeItem{為台灣工程團隊建立與維護 基於 Groovy Script 的 Jenkins CI/CD 流水線。}
    \resumeItem{管理 Bastion 主機與內部資產追蹤系統(Consul + Golang 開發的微服務),並將 EC2 metadata 整合至集中式存取控制機制。}
    \resumeItem{透過觀測性系統(Prometheus / Grafana)分析 系統資源使用情況與效能指標,以識別 基礎設施瓶頸並提出合適的優化方案。}
    \resumeItem{維護觀測性技術堆疊\textbf{(Prometheus、Grafana、Consul)},並支援整體系統監控運作。}
  \resumeItemListEnd

  \resumeSubheading
  {System/Site Reliability Engineering (SRE)/DevOps Engineering}{Nov. 2023 -- Dec. 2024}
  {眾鼎科技有限公司}{}  
  \resumeItemListStart
    \resumeItem{在\textbf{Linux}終端機環境中工作。}
    \resumeItem{參與輪值待命(on-call,約每月 1 週),與團隊成員共同排除正式環境(Prod)中的各類問題。}
    \resumeItem{支援部署於 AWS 或地端基礎設施上的加密貨幣交易所平台。}
    \resumeItem{管理與維運部署於 AWS 或地端環境的 Kubernetes(k8s)叢集。}
    \resumeItem{使用 Pulumi / Ansible 進行 AWS 與 Cloudflare 基礎設施管理。}
    \resumeItem{管理與維運 基於 GitLab CI 或 Jenkins 的 CI/CD 流水線。}
  \resumeItemListEnd

  \resumeSubheading
  {Site Reliability Engineering (SRE)/DevOps/Cloud engineering}{Nov. 2021 -- April. 2023}
  {雲磐科技有限公司}{}  
  \resumeItemListStart
    \resumeItem{在\textbf{Linux}終端機環境中工作。}
    \resumeItem{參與輪值待命(on-call),與團隊成員共同排除各類系統問題。}
    \resumeItem{開發 基於 Python 的網站監控自動化專案。}
    \resumeItem{開發 基於 GCP(GKE)與阿里雲(ACK)的串流平台專案。}
    \resumeItem{透過 JumpServer 管理 雲端基礎設施(AWS/GCP/阿里雲/Cloudflare 等)。}
    \resumeItem{管理與維運部署於 AWS(EKS)/GCP(GKE)/阿里雲(ACK)的 Kubernetes(k8s)叢集。}
    \resumeItem{管理與維運 基於 GitLab CI/Jenkins/Tekton 的 CI/CD 流水線。}
  \resumeItemListEnd

\resumeSubheading
  {物聯網整合開發人才就業班}{Jan. 2021 -- May. 2021}
  {工業技術研究院 產業學院}{}
  \resumeItemListStart
    \resumeItem{Completed an intensive hands-on training program covering \textbf{web development fundamentals, system programming, Linux environments, networking basics, and IoT system integration}.}
    \resumeSubItem{Building an \textbf{IoT monitoring system (IndoorAirBox)} using \textbf{Raspberry Pi and ESP8266}, focusing on temperature and humidity data collection with available course resources.}
  \resumeItemListEnd

  \resumeSubHeadingListEnd  

%-----------PROJECTS-----------

\section{專案}

    \resumeSubHeadingListStart
    \resumeProjectHeading
        {\textbf{homelab}} {Jun. 2023 -- Present}
        \resumeItemListStart
          \resumeItem{建置並維運 基於 \textbf{Proxmox VE} 的基礎設施,以支援多個個人專案的 Linux 實驗環境。}
        \resumeItemListEnd
    \resumeSubHeadingListEnd

    \resumeSubHeadingListStart
    \resumeProjectHeading
        {\textbf{resume}} {Mar. 2023 -- Present}
        \resumeItemListStart
          \resumeItem{建置 基於 \textbf{LaTeX的線上履歷網站},用於記錄我的工作經歷、專案成果、學歷背景與技能。}          
        \resumeItemListEnd          
    \resumeSubHeadingListEnd

%-----------EDUCATION-----------
\section {學歷}
\resumeSubHeadingListStart
\resumeSubheading
  {國立東華大學}{Sep. 2016 -- Jan. 2019}
  {材料科學與工程學系 碩士}{壽豐鄉, 花蓮縣}
    \resumeItemListStart
  \resumeItem {\textbf{論文}: 可層數控制之二硫化鉬薄膜的超快激發-探測瞬間吸收光譜之研究}
    \resumeItem 
        {\textbf{研究主題}: 半導體材料}
    \resumeItemListEnd
\resumeSubHeadingListEnd

\resumeSubHeadingListStart
\resumeSubheading
  {國立東華大學}{Sep. 2012 -- Jun. 2016}
  {材料科學與工程學系 學士}{壽豐鄉, 花蓮縣}
    \resumeItemListStart
    \resumeItem {先進材料學程}
    \resumeItemListEnd
\resumeSubHeadingListEnd

%
%-----------PROGRAMMING SKILLS-----------
\section{技能}
 \begin{itemize}[leftmargin=0in, label={}]
    \small{\item{
     \textbf{語言}{: Mandarin (native), English (professional working proficiency)}\vspace{2pt} \\
     \textbf{程式語言}{: Python(proficient), Bash(competence), SQL(competence), Rust(competence), Golang(beginner)}\vspace{2pt} \\
     \textbf{版本控制}{: git} \\
     \textbf{軟體框架}{: Django, FastAPI, Selenium, pandas, SQLAlchemy} \\
     \textbf{作業系統}{: Linux(proficient), macOS(competence), Windows(beginner)}\\
     \textbf{持續整合(CI) \& 持續部屬(CD)}{:GitLab Runner, Github Actions, Jenkins, Tekton Pipelines, Argo CD} \\
     \textbf{串流 \& 訊息傳遞}{: Apache RocketMQ, RabbitMQ} \\
     \textbf{容器排程 \& 編排}{: Kubernetes(EKS, GKE, ACK, k0s, k3s...etc)} \\
     \textbf{Kubernetes 底下運行的子系統}{: ingress-nginx, cert-manager, ECK(Elastic Cloud on Kubernetes) operator, KEDA, Istio...etc} \\
     \textbf{服務代理}{: NGINX, HAProxy, MetalLB, Envoy} \\     
     \textbf{微服務發現與協調}{: Apache Zookeeper, Nacos, Consul} \\
     \textbf{容器映像檔倉庫}{: Harbor} \\
     \textbf{自動化配置}{: Ansible, Atlantis, Terraform, Terragrunt, Pulumi, Ansible} \\
     \textbf{雲端供應商}{: AWS, Google Cloud, Alibaba Cloud, Cloudflare} \\
     \textbf{專業認證}{: AWS Certified Solutions Architect Associate, AWS Certified Cloud Practitioner}
    }}
 \end{itemize}


%-------------------------------------------
\end{document}